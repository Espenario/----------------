\documentclass [12pt]{article}


\usepackage{ucs}
\usepackage[utf8x]{inputenc} %Поддержка UTF8
\usepackage{cmap} % Улучшенный поиск русских слов в полученном pdf-файле
\usepackage[english,russian]{babel} %Пакет для поддержки русского и английского языка
\usepackage{graphicx} %Поддержка графиков
\usepackage{float} %Поддержка float-графиков
\usepackage[left=20mm,right=15mm, top=20mm,bottom=20mm,bindingoffset=0cm]{geometry}
\usepackage{mathtools} 
\usepackage{setspace,amsmath}
\usepackage{amsmath,amssymb}
\usepackage{dsfont}
\renewcommand{\baselinestretch}{1.2}
 
\usepackage{color} 
\definecolor{deepblue}{rgb}{0,0,0.5}
\definecolor{deepred}{rgb}{0.6,0,0}
\definecolor{deepgreen}{rgb}{0,0.5,0}

\DeclareFixedFont{\ttb}{T1}{txtt}{bx}{n}{12} % for bold
\DeclareFixedFont{\ttm}{T1}{txtt}{m}{n}{12}  % for normal

\usepackage{listings}
 
\lstset{
	language=Python,
	basicstyle=\ttm,
	otherkeywords={self},             % Add keywords here
	keywordstyle=\ttb\color{deepblue},
	emph={MyClass,__init__},          % Custom highlighting
	emphstyle=\ttb\color{deepred},    % Custom highlighting style
	stringstyle=\color{deepgreen},
	frame=tb,                         % Any extra options here
	showstringspaces=false            % 
}
 
\usepackage{hyperref}
 
\hypersetup{
    bookmarks=true,         % show bookmarks bar?
    unicode=false,          % non-Latin characters in Acrobat’s bookmarks
    pdftoolbar=true,        % show Acrobat’s toolbar?
    pdfmenubar=true,        % show Acrobat’s menu?
    pdffitwindow=false,     % window fit to page when opened
    pdfstartview={FitH},    % fits the width of the page to the window
    pdftitle={My title},    % title
    pdfauthor={Author},     % author
    pdfsubject={Subject},   % subject of the document
    pdfcreator={Creator},   % creator of the document
    pdfproducer={Producer}, % producer of the document
    pdfkeywords={keyword1} {key2} {key3}, % list of keywords
    pdfnewwindow=true,      % links in new PDF window
    colorlinks=true,       % false: boxed links; true: colored links
    linkcolor=black,          % color of internal links (change box color with linkbordercolor)
    citecolor=green,        % color of links to bibliography
    filecolor=magenta,      % color of file links
    urlcolor=cyan           % color of external links
}

\title{}
\date{}
\author{}

\begin{document}
\begin{titlepage}
\thispagestyle{empty}
\begin{center}
Федеральное государственное бюджетное образовательное учреждение высшего профессионального образования \\Московский государственный технический университет имени Н.Э. Баумана

\end{center}
\vfill
\centerline{\large{Лабораторная работа №3}}
\centerline{\large{по курсу <<Численные методы>>}}
\centerline{\large{<<Интерполяция B-сплайнами>>}}
\vfill
\hfill\parbox{5cm} {
           Выполнил:\\
           студент группы ИУ9-62Б \hfill \\
           Головкин Дмитрий\hfill \medskip\\
           Проверила:\\
           Домрачева А.Б.\hfill
       }
\centerline{Москва, 2023}
\clearpage
\end{titlepage}

\textsc{\textbf{Цель:}} 

Анализ метода интерполяции функции, основанный на построении кубического в контрольных точках.

\textsc{\textbf{Постановка задачи:}} 

\textbf{Дано:}  Функция $y_i = \phi(x_i),  i = \overline{1,n}$ задана таблично, исходные данные включают ошибки измерения.

\begin{table}[h]
\begin{center}
\begin{tabular}{|c|c|c|c|c|}
\hline
$x_1$ & $x_2$ & ... & $x_{n-1}$ & $x_n$ \\
\hline
$y_1$ & $y_2$ & ... & $y_{n-1}$ & $y_n$ \\
\hline
\end{tabular}
\end{center}
\end{table}

\textbf{Найти:} Функцию (интерполянту) $f(x)$, совпадающую с значениями $y_i, i = \overline{1,n}$ в контрольных точках $x_i, i = \overline{1,n}$:
$$f(x_i)=y_i$$

\textbf{Тестовый пример:} 

Зададим некоторую функцию $\phi(x)$ таблично:

\begin{table}[h]
\begin{center}
\begin{tabular}{|c|c|c|c|c|c|c|c|c|c|c|}
\hline
0.5 & 0.75 & 1.0 & 1.25 & 1.5 & 1.75 & 2.0 & 2.25 & 2.5 & 2.75\\
\hline
2.78 & 2.95 & 3.13 & 3.31 & 3.c51 & 3.72 & 3.94 & 4.18 & 4.43 & 4.69\\
\hline
\end{tabular}
\end{center}
\end{table}

\begin{table}[h]
\begin{center}
\begin{tabular}{|c|c|c|c|c|c|c|c|c|c|c|}
\hline
3.0 & 3.25 & 3.5 & 3.75 & 4.0 & 4.25 & 4.5 & 4.75 & 5.0 & 5.25 & 5.5 \\
\hline
4.97 & 5.27 & 5.58 & 5.92 & 6.27 & 6.65 & 7.04 & 7.47 & 7.91 & 8.38 & 8.89 \\
\hline
\end{tabular}
\end{center}
\end{table}

\textbf{Теоретические сведения:}

Интерполяция, интерполирование — в вычислительной математике способ нахождения промежуточных значений величины по имеющемуся дискретному набору известных значений.

Основная цель интерполяции — получить быстрый (экономичный) алгоритм вычисления значений $y(x)$ для значений $x$, не содержащихся в таблице данных.
Интерполируюшие функции $f(x)$, как правило строятся в виде линейных комбинаций некоторых элементарных функций:
$$f(x)=\sum_{k=0}^N {c_k\Phi_k(x)}$$
где $\{\Phi_k(x)\}$ — фиксированный линейно независимые функции, $c_0, c_1, \cdots, c_n$ — не определенные пока коэффициенты.

Один из способов интерполирования на всем отрезке $[a, b]$ является интерполирование сплайнами.
Сплайном называется кусочно-полиномиальная функция, определенная на отрезке $[a, b]$ и имеющая на этом отрезке некоторое количество непрерывных производных. Преимущества интерполяции сплайнами по сравнению с обычными методами интерполяции – в сходимости и устойчивости вычислительного процесса.

Рассмотрим один из наиболее распространенных в практике случаев – интерполирование функции кубическим сплайном.

Пусть на отрезке $[a, b]$ задана непрерывная функция $y(x)$. Введем разбиение отрезка:
$$ a = x_0 < x_1 < x_2 < \cdots < x_{n-1} < x_n = b $$ и обозначим $y_i=y(x_i),  i = \overline{1,n}$

Сплайном, соответствующим данной функции узлам интерполяции называется функция $s(x)$, удовлетворяющая следующим условиям:
\begin{enumerate}
\item На каждом отрезке $[x_{i-1};x_{i}],  i = \overline{2,n}$ функция $s(x)$ является кубическим многочленом;
\item Функция $s(x)$, а также ее первая и вторая производные непрерывны на отрезке $[a, b]$;
\item $s(x_i)=y_i,  i = \overline{2,n}$ - условие интерполирования.
\end{enumerate}

Сплайн, определяемый условиями 1 – 3, называется интерполяционным кубическим сплайном и имеет вид:
$$S_i(x) = a_i + b_i(x - x_i) + c_i(x - x_i)^2 + d_i(x - x_i)^3$$
Заметим несколько важных условий: $S_i'(x_i) = S_{i+1}'(x_i)$ , $S_i'(x_i) = S_{i+1}'(x_i)$ , $S_{1}''(a) = 0$ , $S_n''(b) = 0$. Последние два условия называют условиями гладкости на краях.
Пусть $n$ - число разбиений на отрезке $[a;b]$. Тогда зададим параметр $h$ следующим образом : $h = \frac{b - a}{n}$

Теперь, используя данные условия, можем получить следующие соотношения:
\begin{enumerate}
\item $d_i = \frac{c_(i+1) - c_i}{3h} , i = \overline{1,n}$;
\item Используя условия гладкости на краях получаем: $c_{1} = 0$ , $c_{n+1} = 0$;
\item $a_i = y_{i-1} , i = \overline{1,n}$;
\item $b_i = \frac{y_i - y_{i-1}}{h} - \frac{h}{3}(c_{i+1} + 2c_i)$;
\end{enumerate}
Следующим шагом необхожимо вычислить коэффициенты $c_i$. Преобразуя далее функцию сплайна, получаем следующую систему:

\begin{equation*}
\begin{cases}
4c_2 + c_3 = \frac{3}{h^2}(y_2 - 2y_1 - y_0) \\
... \\
c_i + 4c_{i+1} + c_{i+2} = \frac{3}{h^2}(y_{i+1} - 2y_i - y_{i-1}) \\
... \\
c_{n-1} + 4c_n = \frac{3}{h^2}(y_n - 2y_{n-1} - y_{n-2}) \\
\end{cases}
\end{equation*}
Данные коэффициенты будем искать с помощью метода прогонки.

\textsc{\textbf{Практическая реализация:}}
Листинг 1. Интерполяция кубическими сплайнами
\begin{lstlisting}[language=python]
#!python
# -*- coding: utf-8 -*-
import numpy as np

def calculate_x_n(a, b, c, d, n):
    #alpha = [0.0] * n
    #beta = [0.0] * n
    alpha = np.zeros(n)
    beta = np.zeros(n)    
    alpha[0] = -c[0] / b[0]
    beta[0] = d[0] / b[0]
    for i in range(1,n-1):
        #print(i)
        alpha[i] = -c[i] / (a[i-1]*alpha[i-1] + b[i])
        beta[i] = (d[i] - a[i-1]*beta[i-1]) / (a[i-1]*alpha[i-1] + b[i])
    x_n = (d[n-1] - a[n-2]*beta[n-2]) / (a[n-2]*alpha[n-2] + b[n-1])
    #print(alpha, beta)
    return x_n, alpha, beta

def calculate_c_n_vector(x_n, alpha, beta, n):
    x = [0.0] * (n-1)
    x.append(x_n)
    for i in range(n-2, -1, -1):
        x[i] = alpha[i]*x[i+1] + beta[i]
    return x

def calculate_d(y_n, h, n):
    d = [y_n[0] / 3 * (h*h)]
    for i in range(2, n+1):
        d.append(y_n[i] - 2*y_n[i-1] + y_n[i-2])
    return (3/(h*h)) * np.asarray(d)


a = 0
b = 1
n = 9
a_coefs = [1.0] * (n-1)
b_coefs = [4.0] * n
c_coefs = [1.0] * (n-1)
x_n = np.arange(a, b + b/18, b/9)
y_n = np.exp(x_n)
print('X: ',x_n)
print('Y: ',y_n)

h = (b - a) / n
d_coefs = calculate_d(y_n, h, n)
#print("---", d_coefs)
spl, alpha, beta = calculate_x_n(a_coefs, b_coefs, c_coefs, d_coefs, n)
#print(spl, alpha, beta)

c_n = calculate_c_n_vector(spl, alpha, beta, n)
c_n.append(0)
c_n = np.asarray(c_n)
print('c_coefs: ', c_n)
d_n = (c_n[1:n+1] - c_n[:n]) / (3*h)
print('d_coefs: ',d_n)
a_n = y_n[:n]
print('a_coefs: ',a_n)
b_n = (y_n[1:n+1] - y_n[:n]) / h - (h / 3) * (c_n[1:n+1] + 2*c_n[:n])
print('b_coefs: ',b_n)

x_n_new = np.arange(a, b + b/9, b/18)
y_n_new = np.exp(x_n_new)
spline_extended = [a_n[(idx-1)//2] + b_n[(idx-1)//2] * (x_n_new[idx] -x_n[(idx-1)//2]) + c_n[(idx-1)//2] * (x_n_new[idx] - x_n[(idx-1)//2])**2 +\
                   d_n[(idx-1)//2] * (x_n_new[idx] - x_n[(idx-1)//2])**3 for idx in range(1, len(a_n)*2+1)]
spline_extended = np.concatenate((a_n[:1], spline_extended))

spline = a_n + b_n*(x_n[1:n+1] - x_n[:n]) + c_n[:n]*(x_n[1:n+1] - x_n[:n])**2 +\
         d_n*(x_n[1:n+1] - x_n[:n])**3
spline = np.concatenate((a_n[:1], spline))

for i in range(0,n+1):
    print("x=", x_n[i], " | f(x)=", y_n[i], " | spl(x)=",spline[i], " | delta=", np.abs(y_n[i]-spline[i]))

for i in range(0,2*n+1):
    print("x=", x_n_new[i], " | f(x)=", y_n_new[i], " | spl(x)=",spline_extended[i], " | delta=", np.abs(y_n_new[i]-spline_extended[i]))


\end{lstlisting}
\textbf{Результаты:}

Для тестирования полученной программы была задана табличная функция:

\begin{table}[h]
\begin{center}
\begin{tabular}{|c|c|c|c|c|c|c|c|c|c|c|}
\hline
0.5 & 0.75 & 1.0 & 1.25 & 1.5 & 1.75 & 2.0 & 2.25 & 2.5 & 2.75\\
\hline
2.78 & 2.95 & 3.13 & 3.31 & 3.c51 & 3.72 & 3.94 & 4.18 & 4.43 & 4.69\\
\hline
\end{tabular}
\end{center}
\end{table}

\begin{table}[h]
\begin{center}
\begin{tabular}{|c|c|c|c|c|c|c|c|c|c|c|}
\hline
3.0 & 3.25 & 3.5 & 3.75 & 4.0 & 4.25 & 4.5 & 4.75 & 5.0 & 5.25 & 5.5 \\
\hline
4.97 & 5.27 & 5.58 & 5.92 & 6.27 & 6.65 & 7.04 & 7.47 & 7.91 & 8.38 & 8.89 \\
\hline
\end{tabular}
\end{center}
\end{table}

В результате работы программы (Листинг 1) получаем значения:

\begin{center}
\begin{tabular}{ |c|c|c|c| }
  \hline
 Значение $x_{n}$ & Значение $y_{n}$ & Значение сплайна $s^{3}(x_n)$ & Разница значений  $s^{3}(x_n)-y_{n}$ \\ \hline
 0.5 & 2.78 & 2.7799999999999994 & -4.440892098500626e-16\\ \hline
 0.75 & 2.95 & 2.952961581335141 & 0.002961581335140906\\ \hline
 1.0 & 3.13 & 3.13 & 0.0\\ \hline
 1.25 & 3.31 & 3.3148652559945737 & 0.004865255994573658\\ \hline
 1.5 & 3.51 & 3.51 & 0.0\\ \hline
 1.75 & 3.72 & 3.7175773946865607 & -0.0024226053134395187\\ \hline
 2.0 & 3.94 & 3.94 & 0.0\\ \hline
 2.25 & 4.18 & 4.178575165259181 & -0.0014248347408187811\\ \hline
 2.5 & 4.43 & 4.43 & 0.0\\ \hline
 2.75 & 4.69 & 4.691871944276711 & 0.0018719442767105576\\ \hline
 3.0 & 4.97 & 4.97 & 0.0\\ \hline
 3.25 & 5.27 & 5.268937057633973 & -0.0010629423660262205\\ \hline
 3.5 & 5.58 & 5.58 & 0.0\\ \hline
 3.75 & 5.92 & 5.901129825187394 & -0.01887017481260589\\ \hline
 4.0 & 6.27 & 6.27 & 0.0\\ \hline
 4.25 & 6.65 & 6.69654364161645 & 0.04654364161644953\\ \hline
 4.5 & 7.04 & 7.039999999999999 & -8.881784197001252e-16\\ \hline
 4.75 & 7.47 & 7.265195608346805 & -0.20480439165319453\\ \hline
 5.0 & 7.91 & 7.909999999999998 & -1.7763568394002505e-15\\ \hline
 5.25 & 8.38 & 9.118923924996324 & 0.7389239249963229\\ \hline
 5.5 & 8.89 & 8.889999999999999 & -1.7763568394002505e-15\\ \hline
\end{tabular}
\end{center}

\textbf{Выводы:}

В ходе выполнения лабораторной работы был рассмотрен метод интерполяции функции, основанный на построении кубического сплайна в контрольных точках, так же для метода была написана реализация на языке программирования Python.

Анализируя результаты полученной программы, можно заметить то, что метод интерполяции функции, основанный на построении кубического в контрольных точках, имеет высокую точность вследствие низкой погрешности. Повышая степень кусочно-интерполяционного многочлена, можно добиться еще лучших результатов аппроксимации функции.

\end{document}
