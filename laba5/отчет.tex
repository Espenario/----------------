\documentclass [12pt]{article}


\usepackage{ucs}
\usepackage[utf8x]{inputenc} %Поддержка UTF8
\usepackage{cmap} % Улучшенный поиск русских слов в полученном pdf-файле
\usepackage[english,russian]{babel} %Пакет для поддержки русского и английского языка
\usepackage{graphicx} %Поддержка графиков
\usepackage{float} %Поддержка float-графиков
\usepackage[left=20mm,right=15mm, top=20mm,bottom=20mm,bindingoffset=0cm]{geometry}
\usepackage{mathtools} 
\usepackage{setspace,amsmath}
\usepackage{amsmath,amssymb}
\usepackage{dsfont}
\renewcommand{\baselinestretch}{1.2}
 
\usepackage{color} 
\definecolor{deepblue}{rgb}{0,0,0.5}
\definecolor{deepred}{rgb}{0.6,0,0}
\definecolor{deepgreen}{rgb}{0,0.5,0}

\DeclareFixedFont{\ttb}{T1}{txtt}{bx}{n}{12} % for bold
\DeclareFixedFont{\ttm}{T1}{txtt}{m}{n}{12}  % for normal

\usepackage{listings}
 
\lstset{
	language=Python,
	basicstyle=\ttm,
	otherkeywords={self},             % Add keywords here
	keywordstyle=\ttb\color{deepblue},
	emph={MyClass,__init__},          % Custom highlighting
	emphstyle=\ttb\color{deepred},    % Custom highlighting style
	stringstyle=\color{deepgreen},
	frame=tb,                         % Any extra options here
	showstringspaces=false            % 
}
 
\usepackage{hyperref}
 
\hypersetup{
    bookmarks=true,         % show bookmarks bar?
    unicode=false,          % non-Latin characters in Acrobat’s bookmarks
    pdftoolbar=true,        % show Acrobat’s toolbar?
    pdfmenubar=true,        % show Acrobat’s menu?
    pdffitwindow=false,     % window fit to page when opened
    pdfstartview={FitH},    % fits the width of the page to the window
    pdftitle={My title},    % title
    pdfauthor={Author},     % author
    pdfsubject={Subject},   % subject of the document
    pdfcreator={Creator},   % creator of the document
    pdfproducer={Producer}, % producer of the document
    pdfkeywords={keyword1} {key2} {key3}, % list of keywords
    pdfnewwindow=true,      % links in new PDF window
    colorlinks=true,       % false: boxed links; true: colored links
    linkcolor=black,          % color of internal links (change box color with linkbordercolor)
    citecolor=green,        % color of links to bibliography
    filecolor=magenta,      % color of file links
    urlcolor=cyan           % color of external links
}

\title{}
\date{}
\author{}

\begin{document}
\begin{titlepage}
\thispagestyle{empty}
\begin{center}
Федеральное государственное бюджетное образовательное учреждение высшего профессионального образования \\Московский государственный технический университет имени Н.Э. Баумана

\end{center}
\vfill
\centerline{\large{Лабораторная работа №5}}
\centerline{\large{по курсу <<Численные методы>>}}
\centerline{\large{<<Метод наименьших квадратов. Аппроксимация алгебраическими многочленами>>}}
\vfill
\hfill\parbox{5cm} {
           Выполнил:\\
           студент группы ИУ9-62Б \hfill \\
           Головкин Дмитрий\hfill \medskip\\
           Проверила:\\
           Домрачева А.Б.\hfill
       }
\centerline{Москва, 2023}
\clearpage
\end{titlepage}

\textsc{\textbf{Цель:}} 

Анализ метода метода наименьших квадратов для решения задачи аппроксимации.

\textsc{\textbf{Постановка задачи:}} 

\textbf{Дано:}  Функция $y_i = \f(x_i),  i = \overline{0,n}$ задана таблично, исходные данные включают ошибки измерения.

\begin{table}[h]
\begin{center}
\begin{tabular}{|c|c|c|c|c|}
\hline
$x_1$ & $x_2$ & ... & $x_{n-1}$ & $x_n$ \\
\hline
$y_1$ & $y_2$ & ... & $y_{n-1}$ & $y_n$ \\
\hline
\end{tabular}
\end{center}
\end{table}

\textbf{Найти:} Гладкую аналитическую функцию  $z(x)$, доставляющую наименьшее значение величине $$ MSE=\sqrt[2]{\sum\limits_{i = 0}^n{(z(x_i) - y_i)^2}}$$ Эту величину называют среднеквадратичным уклонением функции $z(x)$ от системы узлов. Описанный подход к решению задачи приближения функции - методом наименьших квадратов.

\textbf{Тестовый пример:} 

Зададим некоторую функцию $f(x) = y$. Представим значения в виде таблицы:

\begin{table}[h]
\begin{center}
\begin{tabular}{|c|c|c|c|c|c|c|c|c|c|}
\hline
1 & 1.5 & 2 & 2.5 & 3 & 3.5 & 4 & 4.5 & 5\\
\hline
0.16 & 0.68 & 1.96 & 2.79 & 3.80 & 6.81 & 9.50 & 15.60 & 24.86\\
\hline
\end{tabular}
\end{center}
\end{table}

\textbf{Описание метода:}

Как правило, $z(x)$ отыскивают в виде линейной комбинации заданных функций $$ z(x) = \lambda_{1}\phi_{1}(x) + ... + \lambda_m\phi_m(x) $$
Параметры $\lambda_i, i = \overline{1,m}$ являются решениями линейной системы наименьших квадратов $$ A\lambda = b $$,
где $\lambda$ - столбец параметров $\lambda_i$ . $A = (a_ij)$ - симметричная положительно определенная матрица (матрица Грама) с коэффициентами $ a_ij = \sum\limits_{k = 0}^n{\phi_i(x_i)\phi_j(x_k)}$; b - столбец правой части системы, $ b_i = \sum\limits_{k = 0}^n{\phi_i(x_k)y_k}, \quad i,j = \overline{1,m} $


Если приближаемая функция достаточно глабкая, хотя вид ее и неизвестен, аппроксимирующую функцию нередко ищут в виде алгебраического многочлена $$ z(x) = \lambda_{1} + \lambda_{2}x + ... + \lambda_{m}x^(m-1) $$.
Тогда $\phi_i = x^(i-1)$ и элементы матрица Грама получают по формулам $$ a_ij = \sum\limits_{k = 0}^n{x_{k}^(i + j - 2)}$$,
а свободные члены - $$b_i = \sum\limits_{k = 0}^n{y_{k}^(i - 1)}, \quad i, j = \overline{1,m} $$

Абсолютной погрешностью аппроксимации служит среднеквадратичное отклонение (СКО): $$ \Delta = \frac{MSE}{\sqrt[2]{n}} = \frac{1}{\sqrt[2]{n}}\sqrt[2]{\sum\limits_{k = 0}^n{(y_k - \lambda_{1} - \lambda_{2}x_k - ... -\lambda_{m}x_{k}^(m-1))^2}} $$, 
относительная ошибка $$ \delta = \frac{\Delta}{||y||} = \frac{\Delta}{\sqrt[2]{\sum\limits_{k = 0}^n{y_{k}^2}}} $$

\textsc{\textbf{Практическая реализация:}}
Листинг 1. Аппроксимация функции многочленом 3 степени
\begin{lstlisting}[language=python]
#!python
# -*- coding: utf-8 -*-
import numpy as np

def func(x):
    return x**2 + np.exp(x)

x = np.array([1,1.5,2,2.5,3,3.5,4,4.5,5])
y = np.array([0.16, 0.68, 1.96, 2.79, 3.80, 6.81, 9.50, 15.60, 24.86])
y = func(x)
n = len(x)
m = 4

a = np.array([[sum(x**(i + j - 2)) for j in range(1,m+1)] for i in range(1,m+1)])
a

b = np.array([sum(y*x**(i - 1)) for i in range(1,m+1)])
b

lambd = np.dot(np.linalg.inv(a), b)
lambd

x_help = np.array([x**i for i in range(m)])
x_help = np.array([sum(-lambd*[x[j]**i for i in range(m)]) for j in range(n)])
x_help

delta = 1 / np.sqrt(n) * np.sqrt(sum((y + x_help)**2))
delta

sigma = delta / np.sqrt(np.sum(y**2))
sigma

x_inner = (x[:-1] + x[1:]) / 2
z = [sum(lambd*[x_in**i for i in range(m)]) for x_in in x_inner]
z

import matplotlib.pyplot as plt
fig, ax = plt.subplots(figsize= (10,8))
ax.plot(x, y, label='true')
ax.plot(x_inner, z, label='aprox')
ax.legend()

\end{lstlisting}
\textbf{Результаты:}

Для тестирования полученной программы была задана функция $\phi(x) = exp(x)$. Представим её в виде таблицы:

\begin{table}[h]
\begin{center}
\begin{tabular}{|c|c|c|c|c|c|c|c|c|c|c|}
\hline
0.0 & 0.055 & 0.111 & 0.166 & 0.222 & 0.277 & 0.333 & 0.388 & 0.444 & 0.5\\
\hline
1.0 & 1.057 & 1.117 & 1.181 & 1.248 & 1.320 & 1.395 & 1.475 & 1.559 & 1.648\\
\hline
\end{tabular}
\end{center}
\end{table}

\begin{table}[h]
\begin{center}
\begin{tabular}{|c|c|c|c|c|c|c|c|c|}
\hline
0.555 & 0.611 & 0.666 & 0.722 & 0.777 & 0.833 & 0.888 & 0.944 & 1.0 \\
\hline
1.742 & 1.842 & 1.947 & 2.059 & 2.176 & 2.300 & 2.432 & 2.571 & 2.718 \\
\hline
\end{tabular}
\end{center}
\end{table}

В результате работы программы (Листинг 1) получаем значения:

\begin{center}
\begin{tabular}{ |c|c|c|c| }
  \hline
 Значение $x_{n}$ & Значение $y_{n}$ & Значение сплайна $s^{3}(x_n)$ & Разница значений  $s^{3}(x_n)-y_{n}$ \\ \hline
 0.0 & 1.0 & 1.0 & 0.0\\ \hline
 0.055 & 1.057 & 1.0575984147738515 & 0.0004706700136150044\\ \hline
 0.111 & 1.117 & 1.1175190687418637 & 0.0\\ \hline
 0.166 & 1.181 & 1.181233576490805 & 0.00012683637484078858\\ \hline
 0.222 & 1.248 & 1.2488488690016821 & 0.0\\ \hline
 0.277 & 1.320 & 1.3202266534113838 & 3.386497726354243e-05\\ \hline
 0.333 & 1.395 & 1.3956124250860895 & 0.0\\ \hline
 0.388 & 1.475 & 1.475328851159584 & 1.1764331038222053e-05\\ \hline
 0.444 & 1.559 & 1.5596234976067807 & 0.0\\ \hline
 0.5 & 1.648 & 1.6487309531466745 & 9.682446546310786e-06\\ \hline
 0.555 & 1.742 & 1.7429089986334578 & 0.0\\ \hline
 0.611 & 1.842 & 1.8424465712119993 & 3.0887835709814127e-05\\ \hline
 0.666 & 1.947 & 1.9477340410546755 & 2.220446049250313e-16\\ \hline
 0.722 & 2.059 & 2.0591131797740907 & 0.00010948556121936903\\ \hline
 0.777 & 2.176 & 2.176629931716248 & 0.0\\ \hline
 0.833 & 2.300 & 2.3005639380231275 & 0.00041195286969708533\\ \hline
 0.888 & 2.432 & 2.4324254542872077 & 0.0\\ \hline
 0.944 & 2.571 & 2.57291728658255 &  0.0015328517945203401\\ \hline
 1.0 & 2.718 & 2.718281828459045 & 0.0\\ \hline
\end{tabular}
\end{center}

\textbf{Выводы:}

В ходе выполнения лабораторной работы был рассмотрен метод интерполяции функции, основанный на построении кубического сплайна в контрольных точках, так же для метода была написана реализация на языке программирования Python.

Анализируя результаты полученной программы, можно заметить то, что метод интерполяции функции, основанный на построении кубического в контрольных точках, имеет высокую точность вследствие низкой погрешности. Повышая степень кусочно-интерполяционного многочлена, можно добиться еще лучших результатов аппроксимации функции.

\end{document}
