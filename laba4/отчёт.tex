\documentclass [12pt]{article}


\usepackage{ucs}
\usepackage[utf8x]{inputenc} %Поддержка UTF8
\usepackage{cmap} % Улучшенный поиск русских слов в полученном pdf-файле
\usepackage[english,russian]{babel} %Пакет для поддержки русского и английского языка
\usepackage{graphicx} %Поддержка графиков
\usepackage{float} %Поддержка float-графиков
\usepackage[left=20mm,right=15mm, top=20mm,bottom=20mm,bindingoffset=0cm]{geometry}
\usepackage{mathtools} 
\DeclarePairedDelimiter{\abs}{\lvert}{\rvert}
\renewcommand{\baselinestretch}{1.2}
 
\usepackage{color} 
\definecolor{deepblue}{rgb}{0,0,0.5}
\definecolor{deepred}{rgb}{0.6,0,0}
\definecolor{deepgreen}{rgb}{0,0.5,0}
\definecolor{gray}{rgb}{0.5,0.5,0.5}

\DeclareFixedFont{\ttb}{T1}{txtt}{bx}{n}{12} % for bold
\DeclareFixedFont{\ttm}{T1}{txtt}{m}{n}{12}  % for normal

\usepackage{listings}
 
\lstset{
	language=Python,
	basicstyle=\ttm,
	otherkeywords={self},             % Add keywords here
	keywordstyle=\ttb\color{deepblue},
	emph={MyClass,__init__},          % Custom highlighting
	emphstyle=\ttb\color{deepred},    % Custom highlighting style
	stringstyle=\color{deepgreen},
	frame=tb,                         % Any extra options here
	showstringspaces=false            % 
}
 
\usepackage{hyperref}
 
\hypersetup{
    bookmarks=true,         % show bookmarks bar?
    unicode=false,          % non-Latin characters in Acrobat’s bookmarks
    pdftoolbar=true,        % show Acrobat’s toolbar?
    pdfmenubar=true,        % show Acrobat’s menu?
    pdffitwindow=false,     % window fit to page when opened
    pdfstartview={FitH},    % fits the width of the page to the window
    pdftitle={My title},    % title
    pdfauthor={Author},     % author
    pdfsubject={Subject},   % subject of the document
    pdfcreator={Creator},   % creator of the document
    pdfproducer={Producer}, % producer of the document
    pdfkeywords={keyword1} {key2} {key3}, % list of keywords
    pdfnewwindow=true,      % links in new PDF window
    colorlinks=true,       % false: boxed links; true: colored links
    linkcolor=black,          % color of internal links (change box color with linkbordercolor)
    citecolor=green,        % color of links to bibliography
    filecolor=magenta,      % color of file links
    urlcolor=cyan           % color of external links
}

\title{}
\date{}
\author{}

\begin{document}
\begin{titlepage}
\thispagestyle{empty}
\begin{center}
Федеральное государственное бюджетное образовательное учреждение высшего профессионального образования \\Московский государственный технический университет имени Н.Э. Баумана

\end{center}
\vfill
\centerline{\large{Лабораторная работа №4}}
\centerline{\large{по курсу <<Численные методы>>}}
\centerline{\large{<<Численное решение краевой задачи для линейного }}
\centerline{\large{дифференциального уравнения второго порядка методом прогонки и стрельбы>>}}
\vfill
\hfill\parbox{5cm} {
           Выполнил:\\
           студент группы ИУ9-62Б \hfill \\
           Головкин Дмитрий\hfill \medskip\\
           Проверила:\\
           Домрачева А.Б.\hfill
       }
\centerline{Москва, 2023}
\clearpage
\end{titlepage}

\textsc{\textbf{Цель:}} 

Анализ метода прогонки и стрельбы для решения дифференциальных уравнений. 

Решение задачи Коши (частного решения) для дифференциального уравнения 2 порядка с помощью этогих методов.

\textsc{\textbf{Постановка задачи:}}

\textbf{Дано:}  

1) Неоднородное дифференциальное уравнение 2 порядка с постоянными коэффициентами: $$y''(x)+p(x)y'(x)+q(x)y(x)=f(x) \quad(1)$$ где $p(x)$ и $q(x)$ - постоянные функции, а $f(x)$ - функция, непрерывна на интервале интегрирования $[a;b]$.

2) Краевые условия: $$ y(0)=\alpha,\quad y(1)=\beta \quad (2)$$

 \textbf{Найти:} Численное решение данного неоднородного линейного дифференциального уравнения 2 порядка отвечающее краевым условиям методом прогонки и стрельбы.

\textbf{Тестовый пример:} 

В нашем случае, выберем следующие значения:  $$y(x) = e^x, \quad p(x)=1, \quad q(x)=-1$$

Дифференциальное уравнение примет вид:  $$y''(x)+y'(x)-y(x)=e^x$$

Определим краевые условия: $$ y(0)=e^0=1, \quad y(1)=e^1=e, \quad x \in [0;1] $$

\textbf{Описание алгоритма метода прогонки:}

Приближенным численным решением задачи (1), (2) называется сеточная функция $ (X_i, y_i), \: i = 0,...,n$, заданная в $(n+1)-й$ точке $x_i = ih, \: h = 1/n$. 

Обозначим через $p_i = p(x_i), \: q_i = q(x_i), \: f_i = f(x_i)$ значения коэффициентов уравнения в точках $x_i$ (узлах сетки), $i = 0,...,n$. Применяя разностну. аппроксимацию производных по формулам численного дифференцирования, получим приближенную систему уравнений относительно ординат сеточной функции $y_i:$ $$ \frac{y_{i+1} - 2y_i + 2y_{i-1}}{h^2} + p_i\frac{y_{i+1} - y{i-1}}{2h} + q_iy_i=f_i$$ или после преобразований $$ y_{i-1}(1 - \frac{h}{2}p_i) + y_i(h^2q_i - 2) + y_{i+1}(1 + \frac{h}{2}p_i) = h^2f_i, \quad i = 1,...n-1, \: (3) $$ с краевыми условиями $$ y_0 = \alpha, \: y_n = \beta \quad (4)$$

Система (3) является разностной системой с краевыми условиями (4) и представляет собой трехдиагональную систему линейных алгебраических уравнений $(n+1)-го$ порядка. Эту систему будем решать методом прогонки.

\textbf{Описание алгоритма метода стрельбы:}

Рассмотрим сеточный аналог метода стрельбы. Разобьем отрезок $[a,b]$ на n частей точками $x_0, x_1, ... ,X_n$, где $x_i = a + ih, \: h=\frac{b-a}{n}$, а производные (1) во всех внутренних точках заменим разностными аналогами (аналогично подобным расскаждениям в методе прогонки): $$ y'_i = \frac{y_{i+1}-y{i-1}}{2h}, \quad y''_i = \frac{y_{i+1} - 2y_i + y_{i+1}}{h^2}, \quad i = 1,2,...,n-1.$$ Будем искать решения, удовлетворяющие условиям $$y_{0}[0] = A, \quad y_{0}[1] = D_{0}, \quad y_{1}[0] = 0, \quad y_{1}[1] = D_{1} \neq 0, \: (5)$$ 

Используя условия (5) и разностную запись произовдных получем следующие выражения: $$ y_{0}[i+1] = \frac{f_ih^2 + (2 - q_ih^2)y_{0}[i] - (1-p_i\frac{h}{2})y_{0}[i-1]}{1 + p_i\frac{h}{2}},$$ $$ y_{1}[i+1]=\frac{(2 - q_ih^2)y_{1}[i] - (1-p_ih^2)y_{1}[i-1]}{1 + p_i\frac{h}{2}},$$ $$ i = 1,2,...,n-1;$$ $$ y_{0}[0] = A, \quad y_{0}[1] = D_{0}, \quad y_{1}[0] = 0, \quad y_{1}[1] = D_{1}$$

После того, как величины $ y_{0}[2],...,y_{0}[n], \: y_{1}[2],...,y_{1}[n]$ последовательно определены, назодим $C_{1}$ из уравнения $y_{0}[n] + C_{1}y_{1}[n] = B$, т. е. $C_{1} = \frac{B - y_{0}[n]}{y_{1}[n]}$. Искомое решение задачи (1) находим теперь по формулам $$ y[i] = y_{0}[i] + C_{1}y_{1}[i], \quad i = 0,1,...,n$$
\textsc{\textbf{Практическая реализация:}}

Листинг 1. Метод прогонки для решения диф.уравнения
\begin{lstlisting}[language=python]

#!python
# -*- coding: utf-8 -*-

import numpy as np

def calculate_x_n(a, b, c, d, n):
    #alpha = [0.0] * n
    #beta = [0.0] * n
    alpha = np.zeros(n)
    beta = np.zeros(n)    
    alpha[0] = -c[0] / b[0]
    beta[0] = d[0] / b[0]
    for i in range(1,n-1):
        #print(i)
        alpha[i] = -c[i] / (a[i-1]*alpha[i-1] + b[i])
        beta[i] = (d[i] - a[i-1]*beta[i-1]) / (a[i-1]*alpha[i-1] + b[i])
    x_n = (d[n-1] - a[n-2]*beta[n-2]) / (a[n-2]*alpha[n-2] + b[n-1])
    #print(alpha, beta)
    return x_n, alpha, beta

def calculate_x_vector(x_n, alpha, beta, n):
    x = [0.0] * (n-1)
    x.append(x_n)
    for i in range(n-2, -1, -1):
        x[i] = alpha[i]*x[i+1] + beta[i]
    return x

def func(x):
    return np.exp(x)

n = 10
left_b = 0
right_b = 1
h = (right_b - left_b) / n

x = np.arange(left_b, right_b + h/2, h)
y = func(x)

a_coefs = np.array([0.0] +  np.ones(n-2) - (h/2)*np.ones(n-2))
b_coefs = -1 * h**2 * np.ones(n-1) - 2*np.ones(n-1)
c_coefs = np.ones(n-2) + (h/2)*np.ones(n-2)
d_coefs = y[1:-1] * h**2 

y0 = func(left_b)
yn = func(right_b)

d_coefs[0] -= a_coefs[1] * y0
d_coefs[-1] -= c_coefs[0] * yn

print('a_coefs: ',a_coefs)
print('b_coefs: ',b_coefs)
print('c_coefs: ',c_coefs)
print('d_coefs: ',d_coefs)

x_n, alpha, beta = calculate_x_n(a_coefs, b_coefs, c_coefs, d_coefs, n-1)
x = calculate_x_vector(x_n, alpha, beta, n-1)

final_solution = np.concatenate(([y0], x, [yn]))

delta = np.abs(y - final_solution)

for i in range(len(delta)):
    print("y=", y[i], " | apr_sol=", final_solution[i], " | delta=", delta[i])
    
\end{lstlisting}

Листинг 2. Метод стрельбы для решения диф.уравнения

\begin{lstlisting}[language=python]
#!python
# -*- coding: utf-8 -*-

import numpy as np
\end{lstlisting}

\textsc{\textbf{Результаты:}}

Для тестирования полученной программы было выбрано уравнение: $$y''(x)+y'(x)-y(x)=e^x$$
c начальными условиями задачи Коши: $$ y(0)=e^0=1, \quad y(1)=e^1=e, \quad x \in [0;1] $$

Чтобы проверить точность метода, вычислим погрешность: $$\varepsilon=\abs{y^{*}_{n}-y_{n}}$$ где $y^{*}_{n}$ - численное решение данного уравнения, $y_{n}$ - полученное решение посредством метода Рунге-Кутты 4 порядка. 

Ниже приведена таблица результата полученной программы для вычисления погрешности (Листинг 1) на указанном методе:

\begin{center}
\begin{tabular}{ |c|c|c|c| }
  \hline
  Значение $x_{n}$ & Численное решение $y^{*}_{n}$ & Полученное решение $y_{n}$ & Погрешность $\varepsilon$ \\ \hline
  0 & 1 & 1.0 & 0.0 \\ \hline
  0.1 & 1.1051707977298888 & 1.1053447081380232 & 0.00017379006237550065 \\ \hline
  0.2 & 1.2214025180676706 & 1.2217091167030314 & 0.0003063585428615401 \\ \hline
  0.3 & 1.349858446117391 & 1.3502589327843548 & 0.00040012520835164267 \\ \hline
  0.4 & 1.4918242110670952 & 1.4922813162899364 & 0.00045661864866608504 \\ \hline
  0.5 & 1.6487206531791918 & 1.6491978157848073 & 0.00047654508467909196 \\ \hline
  0.6 & 1.8221180440212896 & 1.8225786399609751 & 0.0004598395704660252 \\ \hline
  0.7 & 2.0137518022579552 & 2.014158408885617 & 0.0004057014151404026 \\ \hline
  0.8 & 2.2255398622919795 & 2.225853543782731 & 0.00031261529026327395 \\ \hline
  0.9 & 2.459601869586057 & 2.4597814703303595 & 0.00017835917340969232 \\ \hline
  1.0 & 2.7182803947778686 & 2.718281828459045 & 0.0 \\ \hline
\end{tabular}
\end{center}

Ниже приведена таблица результата полученной программы для вычисления погрешности (Листинг 2) на указанном методе:

\begin{center}
\begin{tabular}{ |c|c|c|c| }
  \hline
  Значение $x_{n}$ & Численное решение $y^{*}_{n}$ & Полученное решение $y_{n}$ & Погрешность $\varepsilon$ \\ \hline
  0 & 1 & 1.0 & 0.0 \\ \hline
  0.1 & 1.1051707977298888 & 1.1053447081380232 & 0.00017379006237550065 \\ \hline
  0.2 & 1.2214025180676706 & 1.2217091167030314 & 0.0003063585428615401 \\ \hline
  0.3 & 1.349858446117391 & 1.3502589327843548 & 0.00040012520835164267 \\ \hline
  0.4 & 1.4918242110670952 & 1.4922813162899364 & 0.00045661864866608504 \\ \hline
  0.5 & 1.6487206531791918 & 1.6491978157848073 & 0.00047654508467909196 \\ \hline
  0.6 & 1.8221180440212896 & 1.8225786399609751 & 0.0004598395704660252 \\ \hline
  0.7 & 2.0137518022579552 & 2.014158408885617 & 0.0004057014151404026 \\ \hline
  0.8 & 2.2255398622919795 & 2.225853543782731 & 0.00031261529026327395 \\ \hline
  0.9 & 2.459601869586057 & 2.4597814703303595 & 0.00017835917340969232 \\ \hline
  1.0 & 2.7182803947778686 & 2.718281828459045 & 0.0 \\ \hline
\end{tabular}
\end{center}

\textsc{\textbf{Выводы:}}

В ходе выполнения лабораторной работы были рассмотрены 2 метода решений дифференциальных уравнений, а именно методы прогонки и стрельбы. Была написана реализация данных методов на языке программирования Python.

\end{document}